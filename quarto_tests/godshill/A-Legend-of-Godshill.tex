% Options for packages loaded elsewhere
\PassOptionsToPackage{unicode}{hyperref}
\PassOptionsToPackage{hyphens}{url}
%
\documentclass[
  12pt,
  a5paper,
  twoside]{book}

\usepackage{amsmath,amssymb}
\usepackage{iftex}
\ifPDFTeX
  \usepackage[T1]{fontenc}
  \usepackage[utf8]{inputenc}
  \usepackage{textcomp} % provide euro and other symbols
\else % if luatex or xetex
  \usepackage{unicode-math}
  \defaultfontfeatures{Scale=MatchLowercase}
  \defaultfontfeatures[\rmfamily]{Ligatures=TeX,Scale=1}
\fi
\usepackage{lmodern}
\ifPDFTeX\else  
    % xetex/luatex font selection
    \setmainfont[]{Times New Roman}
\fi
% Use upquote if available, for straight quotes in verbatim environments
\IfFileExists{upquote.sty}{\usepackage{upquote}}{}
\IfFileExists{microtype.sty}{% use microtype if available
  \usepackage[]{microtype}
  \UseMicrotypeSet[protrusion]{basicmath} % disable protrusion for tt fonts
}{}
\makeatletter
\@ifundefined{KOMAClassName}{% if non-KOMA class
  \IfFileExists{parskip.sty}{%
    \usepackage{parskip}
  }{% else
    \setlength{\parindent}{0pt}
    \setlength{\parskip}{6pt plus 2pt minus 1pt}}
}{% if KOMA class
  \KOMAoptions{parskip=half}}
\makeatother
\usepackage{xcolor}
\usepackage[inner=0.8in,outer=0.5in,top=0.9in,bottom=0.9in]{geometry}
\setlength{\emergencystretch}{3em} % prevent overfull lines
\setcounter{secnumdepth}{5}
% Make \paragraph and \subparagraph free-standing
\makeatletter
\ifx\paragraph\undefined\else
  \let\oldparagraph\paragraph
  \renewcommand{\paragraph}{
    \@ifstar
      \xxxParagraphStar
      \xxxParagraphNoStar
  }
  \newcommand{\xxxParagraphStar}[1]{\oldparagraph*{#1}\mbox{}}
  \newcommand{\xxxParagraphNoStar}[1]{\oldparagraph{#1}\mbox{}}
\fi
\ifx\subparagraph\undefined\else
  \let\oldsubparagraph\subparagraph
  \renewcommand{\subparagraph}{
    \@ifstar
      \xxxSubParagraphStar
      \xxxSubParagraphNoStar
  }
  \newcommand{\xxxSubParagraphStar}[1]{\oldsubparagraph*{#1}\mbox{}}
  \newcommand{\xxxSubParagraphNoStar}[1]{\oldsubparagraph{#1}\mbox{}}
\fi
\makeatother


\providecommand{\tightlist}{%
  \setlength{\itemsep}{0pt}\setlength{\parskip}{0pt}}\usepackage{longtable,booktabs,array}
\usepackage{calc} % for calculating minipage widths
% Correct order of tables after \paragraph or \subparagraph
\usepackage{etoolbox}
\makeatletter
\patchcmd\longtable{\par}{\if@noskipsec\mbox{}\fi\par}{}{}
\makeatother
% Allow footnotes in longtable head/foot
\IfFileExists{footnotehyper.sty}{\usepackage{footnotehyper}}{\usepackage{footnote}}
\makesavenoteenv{longtable}
\usepackage{graphicx}
\makeatletter
\newsavebox\pandoc@box
\newcommand*\pandocbounded[1]{% scales image to fit in text height/width
  \sbox\pandoc@box{#1}%
  \Gscale@div\@tempa{\textheight}{\dimexpr\ht\pandoc@box+\dp\pandoc@box\relax}%
  \Gscale@div\@tempb{\linewidth}{\wd\pandoc@box}%
  \ifdim\@tempb\p@<\@tempa\p@\let\@tempa\@tempb\fi% select the smaller of both
  \ifdim\@tempa\p@<\p@\scalebox{\@tempa}{\usebox\pandoc@box}%
  \else\usebox{\pandoc@box}%
  \fi%
}
% Set default figure placement to htbp
\def\fps@figure{htbp}
\makeatother

% Shared LaTeX styling for book projects
% Save as: island-tales-shared-book-style.tex

% Document initialization (equivalent to your \AtBeginDocument block)
\usepackage{fncychap}
\usepackage{geometry}
\AtBeginDocument{
  \date{}
  \setcounter{tocdepth}{0}
  \setcounter{secnumdepth}{0}
  %\newgeometry{twoside,inner=0.9in,outer=0.5in,top=1in,bottom=1in}
  
  % The \headsep sets the distance between header line and body text
  \setlength{\headsep}{16pt}

  \clubpenalty=10000
  \widowpenalty=100000
  \displaywidowpenalty=10000
  \tolerance=1000
  \hyphenpenalty=50
  \exhyphenpenalty=50

  \let\origtableofcontents\tableofcontents
  \renewcommand{\tableofcontents}{%
    \cleardoublepage
    \pagestyle{plain}
    \origtableofcontents
    \cleardoublepage
    \pagestyle{normal} 
  }
}

% Chapter formatting
\usepackage{titlesec}
\titleformat{\chapter}[display]
  {\normalfont\huge\bfseries\hyphenpenalty=10000\exhyphenpenalty=10000}
  {}
  {0pt}
  {\Huge}
\titlespacing*{\chapter}{0pt}{-10pt}{20pt}

% Code font size (Sphinx compatibility)
%\renewcommand{\sphinxcode}[1]{\texttt{\footnotesize #1}}
%\renewcommand{\sphinxupquote}[1]{#1}
\let\oldtexttt\texttt        % save original \texttt
\renewcommand{\texttt}[1]{{\footnotesize\oldtexttt{#1}}}

% Custom page styles with fancyhdr
\usepackage{fancyhdr}
\makeatletter
\fancypagestyle{normal}{
    \fancyhf{}
    \fancyhead[RE]{{\small\textit{\@title}}}
    \fancyhead[LO]{{\small\nouppercase{\textit{\leftmark}}}}
    \fancyhead[LE]{{\small\thepage}}
    \fancyhead[RO]{{\small\thepage}}
    \fancyfoot[LE,LO,RE,RO]{}
    \setlength{\footskip}{0pt}
    \addtolength{\textheight}{20pt}
    \renewcommand{\headrulewidth}{0.4pt}
}
\makeatother


\fancypagestyle{plain}{
    % first page of new chapter
    \fancyhf{}
    \fancyfoot[LE,LO,RE,RO]{}
    \setlength{\footskip}{0pt}
    \addtolength{\textheight}{20pt}
    \renewcommand{\headrulewidth}{0pt}
}

% New header
%\usepackage{titling}
\usepackage{hyperref}
\newfontfamily\headingfont[]{Gill Sans}


\makeatletter
\renewcommand{\maketitle}{%
  \thispagestyle{plain}
  \noindent\rule{\linewidth}{1pt}\par
  \begingroup
    \hypersetup{pdfauthor={\@author}, pdftitle={\@title}}%
  \endgroup
  \begin{flushright}
    \vspace{48pt}
    {\headingfont
    {\fontsize{44pt}{56pt}\selectfont \setlength{\baselineskip}{56pt}%
    \@title\par}
    \vspace{96pt}
    % HACK to handle quote
    %{\Large \@author}\par
    {\Large Tony {\textquotesingle}Monty{\textquotesingle} Hirst}\par
    
    \vspace{25pt}
    \@date \par

    }
  \end{flushright}

  
  \newpage
  \thispagestyle{plain}
  \thispagestyle{empty}

  \makeatletter

  \vspace*{\fill}

  \begin{center}
  %{\Large \textbf{\@title}}\\[1em]
  {\large \textbf{\textit{A Storynotes Collection}}}\\[2em]
  \end{center}

  \vfill

  \begin{center}
  \large \textit{montystoryteller.org}
  \end{center}

  \vspace{2cm}

  \begin{center}
  \textit{Copyright {\copyright} Anthony Hirst, 2025.\linebreak\linebreak This work may include transcripts of nineteenth century newspaper articles contemporary to the events described, sourced from the British Newspaper Archive, and transcribed by the author.\linebreak\linebreak This work may include transcripts of out of copyright works published prior to 1920 sourced from the Internet Archive.\linebreak\linebreak This work may include other openly licensed materials, acknowledged at the point of use.
  }
  \end{center}

  \vspace*{\fill}

  \begin{center}
    \textit{A Storynotes publication from montystoryteller.org}
  \end{center}
  \makeatother

  \setcounter{footnote}{0}
  \let\thanks\relax\let\maketitle\relax
}
\makeatother

\renewcommand{\chaptermark}[1]{%
  \markboth{#1}{}}

\usepackage{caption}
\captionsetup[figure]{labelformat=empty}

% Apply normal page style (equivalent to fancy in Quarto)
\pagestyle{normal}
\makeatletter
\@ifpackageloaded{bookmark}{}{\usepackage{bookmark}}
\makeatother
\makeatletter
\@ifpackageloaded{caption}{}{\usepackage{caption}}
\AtBeginDocument{%
\ifdefined\contentsname
  \renewcommand*\contentsname{Table of contents}
\else
  \newcommand\contentsname{Table of contents}
\fi
\ifdefined\listfigurename
  \renewcommand*\listfigurename{List of Figures}
\else
  \newcommand\listfigurename{List of Figures}
\fi
\ifdefined\listtablename
  \renewcommand*\listtablename{List of Tables}
\else
  \newcommand\listtablename{List of Tables}
\fi
\ifdefined\figurename
  \renewcommand*\figurename{Figure}
\else
  \newcommand\figurename{Figure}
\fi
\ifdefined\tablename
  \renewcommand*\tablename{Table}
\else
  \newcommand\tablename{Table}
\fi
}
\@ifpackageloaded{float}{}{\usepackage{float}}
\floatstyle{ruled}
\@ifundefined{c@chapter}{\newfloat{codelisting}{h}{lop}}{\newfloat{codelisting}{h}{lop}[chapter]}
\floatname{codelisting}{Listing}
\newcommand*\listoflistings{\listof{codelisting}{List of Listings}}
\makeatother
\makeatletter
\makeatother
\makeatletter
\@ifpackageloaded{caption}{}{\usepackage{caption}}
\@ifpackageloaded{subcaption}{}{\usepackage{subcaption}}
\makeatother

\usepackage{bookmark}

\IfFileExists{xurl.sty}{\usepackage{xurl}}{} % add URL line breaks if available
\urlstyle{same} % disable monospaced font for URLs
\hypersetup{
  pdftitle={A Legend of Godshill},
  pdfauthor={Tony `Monty' Hirst},
  hidelinks,
  pdfcreator={LaTeX via pandoc}}


\title{A Legend of Godshill}
\author{Tony `Monty' Hirst}
\date{}

\begin{document}
\frontmatter
\maketitle

\renewcommand*\contentsname{Contents}
{
\setcounter{tocdepth}{0}
\tableofcontents
}

\mainmatter
\bookmarksetup{startatroot}

\chapter{Preface}\label{preface}

When approaching the village of Godshill, on the Isle of Wight, using
the road from nearby Shanklin, you might not realise there is a Church
in the village at all. If making your way up from Whitwell, a glance
across the fields will reveal the Church arising above what looks like a
small copse of trees, before it disappears out of sight again. And for
visitors arriving via the Newport road will catch a glimpse of the
Church ahead of them, before losing sight of it altogether as they enter
the village.

If you do manage to escape the clutches of the tea rooms and the
miniature village, and make your way up the hill past what was in recent
times a chocolatier's shop, you will find a much quieter old village,
and the Church itself.

How the Church came to be up there on the top of the hill is a matter of
legend, and this short volume brings together several different tellings
of it, pulled from the archives: a tall tale from the pseudonymous
Abraham Elder (1839), a poetical version from Percy G. Stone (1912), and
several tourist guide versions dating back to the nineteenth century.

The volume also includes a contemporary retelling based on Abraham
Elder's tall tale that I have told as part of the \emph{'Tis Tales} set,
\emph{Island Tales}, and a brief background note regarding the possible
identity of Abraham Elder.

As with many local tales and legends, they are best served told. For the
traditional author of written works, the story, as much as the published
work, whether a printed book, a digital e-book, an audio book or even
just an extract in a Sunday supplement, is often something to be
protected, something that remains under the ownership of the author to
license and ``perform'' as they see fit. But for the traditional
storyteller, the product \emph{is} the performance. The stories are
\emph{passed on}, free to be retold by those who heard them, under
common, public ownership. Whilst in one sense this volume is a written
collection that can be read, my greater hope is that it provides you
with a tale that you can tell on, perhaps even as you visit the village
of Godshill itself.

--Tony ``Monty'' Hirst\\
Apse Heath, September 2022.

\let\thefootnote\relax\footnotetext{To keep up to date with traditional storytelling events and activities on the Isle of Wight, make sure you check the `Tis Tales website — tistales.org.uk — regularly. For more storynotes publications, please visit montystoryteller.org}

\bookmarksetup{startatroot}

\chapter{Legends of the Isle of Wight:
Godshill}\label{legends-of-the-isle-of-wight-godshill}

\textbf{\emph{As originally told by Abraham Elder, in
\href{https://archive.org/details/sim_bentleys-miscellany_1839-07_6/page/255/mode/1up?q=godshill}{\emph{Bentley's
Miscellany}, volume 6, 1839}. The tale can also be found in the second
edition of Elder's ``Tales and Legends of the Isle of Wight'', published
in 1843, copies of which are available for inspection by appointment in
the Castle Museum library at Carisbrooke Castle.}}

Having received a letter from Captain Nosered, of Violet Cottage,
Ventnor, containing an invitation for Mr.~Winterblossom and myself to
spend the day with him, stating at the same time that he had a tale for
me connected with that neighbourhood, very curious, and
well-authenticated, which he wished to show me; as the captain was an
old friend of mine, we accepted the invitation, and set out in a car
together the next day.

``Pray, sir,'' said I, as we went along,``what is that church that I see
yonder perched up at the top of a hill?''

``Godshill,'' answered the antiquary.

'' Godshill! Pray can you inform me how it got that name? It cannot be
because it is nearer to heaven than the country round it.''

``I certainly never heard that reason for it before. I always understood
that it had been named Godshill in commemoration of a miracle that
tradition tells us was performed at the building of the church. The
story, as it is now told, and by many still believed, in the Isle of
Wight, is as follows:---

``A sum of money having been given by certain pious individuals, whose
names unfortunately are now lost, for the erection of a church, the
religious authorities of the Island, under whose direction it was to be
erected, looked out for a proper site for it. After mature deliberation,
they fixed upon a spot at the foot of the steep eminence upon which the
present church stands.

``Having arranged this to their own satisfaction, they sent a messenger
to the proprietor of the land, informing him that the Bishop of the Isle
of Wight, after a solemn consultation with a council composed of ancient
and holy men, having at heart the spiritual welfare of his island flock,
had at length decided upon conferring upon him the high honour and
distinction of allowing the church to be built upon his land; and he
begged him moreover not to be puffed up with pride thereat, but to
receive the favour thus conferred upon him with all humility and
gratitude.

``Now it so happened that the owner of this land was a poor franklin (a
freeholder), of very limited means and a very large family, and moreover
he was by no means of a religious turn of mind. In his heart he hated
all priests and monks; he went to sleep at mass when he did attend it;
fast-day and feast were to him alike; and as for confession, he avoided
it altogether, --- not because he had nothing to confess, but because he
was afraid of frightening the priest if he told the truth; and where was
the good of confession if he told lies.

``There were, however, occasional exceptions to this rule. There was a
certain jolly wandering friar, who used to visit him occasionally and
shrive him, without being too particular about trifles; and, besides, he
used to hear his confession after supper, which tended to make it pass
off very smoothly. Once, indeed, the friar ordered him a slight penance;
but then upon that visit he found his landlord's ale a little turned,
which might in some degree have soured his temper. The franklin used to
say, that a simmering mug of ale, with a roasted crab bobbing about in
it, would get him absolution from any sin in the world.

``This being the character of the man who owned the land, it may easily
be imagined that, although he avoided the first evil of being conferred
upon him with all the humility and gratitude required of him.

``He did not, however, dare to fly in the face of his powerful
self-styled benefactors. He hemmed, and hawed, and coughed, and then
remarked what a splendid site for the church there was just at the top
of the hill.

``He was informed that that situation had been well considered, and it
was thought to be too much exposed.

``The franklin then changed his tone, and, looking down to the ground
with a well-feigned humility, he said to the monk ---

``\,`Father, the fact is, I am a very great sinner; and if the church is
built upon land belonging to me, it will be erected upon unholy ground.
I pray you, father, consider this well. My neighbours on both sides are
pious persons, and their land contains magnificent sites for building
churches. If you build your church upon their land, it will not stand
upon unholy ground; and the high honour will be conferred upon a pious
person, who is worthy to be distinguished by the favour, of the bishop
and his reverend council.'

``The monk replied, `Your being a sinner is no obstacle, but the
reverse; for, when the foundation stone is laid, you will receive
absolution for all your sins, be they ever so black; and as for the land
being tainted with unholiness, we can consecrate that.'

``The franklin was now sorely puzzled what to say. He muttered something
about the largeness of his family and the smallness of his farm, and how
the spot fixed upon was the best bit of the whole, and how he might be
reduced to poverty.

``The monk, however, turned a deaf ear to all this, affecting either not
to hear or not to understand the drift of his argument; and so, without
in the least committing himself by any hint about the possibility of
compensation, he hied him back to his masters, and told them how, when
he had delivered his message, the franklin bent his eyes with all
humility towards the ground, and replied, that he was too great a sinner
for so high an honour to be conferred upon him.

``In the due course of time the bishop's architect came to survey the
spot, and trace out the lines of the foundation, and some stones from
the quarry at Binstead were piled in a heap, ready for the commencement
of the building. The next morning the architect and the masons made
their appearance. How great was their astonishment to find not a single
stone remaining where they had placed it, and not a single peg or mark
put in by the architect remaining there!

``They stood here for some time, first staring at the bare field, then
looking at one another, and then staring at the ground again.

``\,`Where are all the building stones gone to?' said one.

``\,`Where are all my pegs that marked out the lines of the foundation?'
said the architect.

``\,`Where are all the stones and the pegs gone to, Master Franklin?
What tricks have you been playing us, Master Franklin?' said one of them
to the owner of the field.

``The franklin looked innocence itself, then opened his eyes and his
mouth, and raised up his hands in mute astonishment.

``\,`It strikes me' said one of the labourers, scratching his head,
`that we most just have mistaken our way, and come to the wrong field.'

``\,`That's quite impossible!' said two or three of the others, speaking
together.

``While they were thus debating, the owner of the land at the top of the
hill made his appearance among them.

``\,`Is this fair? -- is this right? --- is this honourable?' said he.

``\,`What fair?--- what right?' rejoined the architect. `We do not
understand you.'

``\,`I know well,' said the man from the top of the hill, `that land is
oftentimes seized to erect a church upon, without compensation being
given to the owner; but I ask you, is it not hard, very hard, that the
foundations of a church should be pegged out, and the stones placed
ready for the builder, upon my land, without my being told a word about
it beforehand? Sir, I honour the priesthood and holy men as a good man
ought; but not when they come like a thief in the night to plunder me of
my patrimony. Fie! fie! Master Architect. What! --- must you come in the
night, while I am asleep, to mark out your foundations, and place your
building-stones all ready to begin with? Why, if I had overslept myself,
I might almost have found when I awoke my best field converted into
buildings and churchyards.'

``\,`What can the man mean?' said the architect, when the little man
from the top of the hill stopped to take breath.

``\,`Why, it is just what I thought,' said one of the masons; `there
must be two fields somehow or other so exactly alike, that we must have
mistaken the one for the other.'

``\,`I can assure you,' said our friend the franklin, putting in his
word, `that, although he appears a little excited at present, he is a
very sensible, respectable, pious man; but what he is talking about I
cannot imagine.'

``\,`Look up there,' said the little man from the top of the hill;
`there they have already brought stones to commence a church with, and
have actually begun to mark out the direction of the foundations.'

``In consequence, everybody did look up in the direction he pointed, and
certainly they did perceive the tops of two heaps of stones showing
themselves above the brow of the hill. The architect and his assistants
immediately directed their steps there, and, to their great
astonishment, they found the building-stones disposed in much the same
order on the top of the hill that they had placed them in the field
below.

``What was to be done? The bishop had arranged that he should come that
very afternoon to lay the first stone of the church himself. There was,
therefore, no time to be lost; so, without speculating further how the
stones had contrived to get up to the top of a steep hill without
assistance, they set themselves to work in good earnest to bring them
down again; and before the appointed time for the bishop's arrival the
stones were all heaped up as they were before, the architect had pegged
out the shape of the new church, and a little part of the foundation had
been dug, ready to receive the first stone.

``Shortly after the hour at which the bishop was expected, a group of
monks and other ecclesiastics were seen collected together in the
distance waiting for him. After the lapse of about twenty minutes, the
dignitary himself riding on a mule, attended by about six or seven
mounted attendants, joined their inferior brethren, who were awaiting
him. They now formed themselves into a procession, walking two and two,
those on foot marching first, then the bishop; his mounted companions
followed two and two, and a few more attendants on foot brought up the
rear.

``As they advanced at a slow pace, they chanted a psalm. One half of
them chanted the first verse, the other half replied to them in a higher
note, while here and there their united voices swelled into a loud
chorus.

``The workmen and the peasantry, who were assembled round the destined
site of the new church, listened with deep devotion to the solemn notes
of the holy song, now swelling loud, now dying away upon the summer
wind.

``When the procession arrived at the spot, the monks on foot filed to
the right and to the left, still raising their voices, and turning up
their eyes towards heaven. The bishop on his mule now arrived in front,
and it was expected that he would dismount and offer up a prayer for the
success of their undertaking. Had he been on foot, there is no doubt but
that he would have done so; but mules are animals proverbially
obstinate, delighting in showing that they have a will of their own,
independent of their master's. So was it in the present instance; for
the animal, instead of stopping short, as he was directed to do,
continued to walk leisurely on, till at length he quickened his pace
into a trot, and he had actually ascended half way up the steep hill in
front before he could be brought to a full stop. At length the bishop
returned crestfallen and out of humour, and having taken his appointed
place, he commenced his prayer for the success of the undertaking,
resting his knee upon an embroidered footstool, while the rest of the
congregation knelt upon the ground. After his prayer was concluded, some
masonic tools and a small silver coin were given to him. He now, with
the assistance of two masons, deposited the coin, and settled down the
stone upon it. They chanted a psalm; and when this was concluded, the
bishop's attendant deacon called for the franklin by name. When he had
come, the bishop said, `Kneel down.'

``The franklin knelt.

``The bishop then, after praising him for his piety, pronounced a full
absolution for all his sins, and all the ecclesiastics responded in a
deep `Amen.' The bishop then gave the whole assembly his parting
benediction, and the ceremony was at an end.

``As the venerable fathers rode home together, they discussed and
rediscussed, and commented upon the curious tale of which they had heard
several versions that morning; how all the building-stones, together
with the architect's markers and pegs, had been mysteriously conveyed
away from their allotted spot to the top of a steep hill in the
neighbourhood. It could not have been chance. If the stones had rolled
from the top of the hill down to the bottom, it would have been another
thing; but stones cannot roll up a hill.

``Was it a miracle? Catholic priests in all ages of the world are
supposed to be oftener preachers than believers of the miracles that
take place under their own eyes; so, though the possibility of its
having been a miracle was thrown out once or twice, the majority were
decidedly against the opinion that a miracle had been worked in the
present instance.

``Then there was a third supposition. It might have been a trick played
upon them by some base reprobate. This appeared to them all to be much
more unlikely than either of the two foregoing suppositions. Where could
a man be found so utterly wicked as to wish to do such an action?
Certainly not in the Isle of Wight, so celebrated for its piety. But
even suppose such a man was found, how was it possible to imagine for a
moment that he would dare to do it? The church can excommunicate as well
as bless; besides, people had been burnt alive for sacrilege before;
then what object could any person possibly have in doing so? It
certainly could not be merely for the sake of running the chance of
being burnt alive, with the addition of the curses of the church, and
the execration of all mankind. Then, again, how could he possibly carry
his intentions into execution, even if he was mad enough to desire it?
It could have been no light labour to have carried all the stones up the
hill; and it was evidently quite impossible to have done it without
being observed by some of the neighbours; and what neighbour would dare
to conceal such an action from the Holy Church?

``At length one of the brothers interrupted this discussion, saying, in
a most solemn tone,

``\,`In the blindness of your hearts, and in the eagerness of your
talking, yon have altogether forgotten the most important fact of all.'

``\,`What is that?' demanded two or three at once.

``\,`Had it not been for the assistance of two strong men in stopping
his mule, the bishop himself would have been carried up to the top of
the hill.'

``It would never have done for the other ecclesiastics to have cast any
reflections upon the horsemanship of their superior; so it was
absolutely necessary for them all to come to the conclusion that there
was something very supernatural and wonderful in the whole affair. Thus
ostensibly, at any rate, the theory of the miracle carried it hollow.

``The bishop, however, between whom and the mule similar differences of
opinion, attended with precisely the same results, had frequently
occurred before, could not in his heart subscribe to the proof that
appeared to have convinced the rest; so he thus addressed his
attendants.

``\,`Brethren, however singular may have appeared what we have heard and
seen this day, we ought not lightly to adopt an opinion that anything
has occurred out of the common order of nature, lest other causes,
simple and obvious to the unlearned, should by chance be brought to
light, sufficient to account for what has happened, and thus the
authority of the Church be brought into jeopardy. I will therefore order
two men to be placed to watch the spot to-night, and to-morrow we will
discuss this matter again, after they should have made their report.'

``One of his attendants was in consequence sent back to direct two of
the workmen to remain on the spot all night, and to give them his
blessing, which was accordingly done.

``A messenger from the bishop was sent to them again in the morning, to
see whether all had remained quiet during the night. The account that he
brought beck was, that he found the two men lying upon the ground in a
helpless state, like men weary in body and oppressed with strong drink.
He roused them with some trouble, and they then gave a very strange and
marvellous account of what they had seen and heard during the night.

``The most extraordinary fact, however, that the messenger had to report
was, that the stones had all contrived to get up to the top of this hill
again; the foundation-stone had been taken away, and the trench filled
up, and the turf laid smooth again.

``Upon ascending the hill, they found the building-stones bestowed in
the same form they were the morning before; the lines of the foundation
were in the same manner pegged out by the architect's marks; a small
portion of the foundation had been dug, and the first stone had been
laid, --- the identical first stone that had been laid by the bishop in
another place the evening before.

``The bishop, upon hearing this, ordered the two watchers and all the
other persons who had been employed the day previous to be brought
before him. The account that the two watchers gave was, that about
midnight they were startled by a low rumbling noise, which appeared to
issue from the heaps of stones. Presently the stones were observed to
move, rolling about one against another, just as if there was a large
body moving about and kicking in the midst of the heap; then a little
stone rolled off the top of the heap, and tumbled on the ground; but it
quite made their hair stand on end to see that, instead of stopping
there, it kept on rolling and rolling, ---where the ground was rough it
hopped and skipped, and then went on rolling again in the direction of
the hill. Then out came another stone, and rolled, and skipped, and
rolled like the first. In a little time, when the stones had contrived
to shake themselves out of the heap, where they seemed to be very much
in one another's way, they all began rolling away together, --- the
little ones going faster and more nimbly than the others. The watchers
said that they had some difficulty in getting out of the way, there were
so many of them on the move together. A large stone, indeed, did come
foul of one of them, hit him on the shin, and knocked him out of the
way, nearly breaking his leg, and then went bowling on, as if it did not
care whether his leg was broken or not.

``When the stones had all gone by, they determined, though they were
very much frightened at the time, to follow them, and see what they
would do. They overtook them at a steep pitch of the hill, which
appeared to offer considerable hindrance to their ascent The little
ones, indeed, were seen scrambling up without any very great difficulty;
but the large heavy ones could hardly get on at all. Some of them rolled
half way over, and then rolled back again, but, after one or two
efforts, they generally got a roll in advance; and when they passed the
steep pitch, they bowled away again merrily.

``The watchers waited until they had all passed the difficulty except
one large stone, with a very awkward angle sticking out of its side,
which seemed effectually to prevent its turning over at all. It
contrived to turn half way over, and then rolled back again, and this it
had repeated so often, that it had actually worked itself into a hole,
and all its efforts to extricate itself seemed hopeless.

``The watchers consulted with one another whether it would not be an act
of charity to lend the poor stone a hand, and then they knelt down and
put their shoulders against its under side and gave a heave. The great
awkward stone rolled over, and then kept scrambling on as if it had been
just as well made as the rest of its companions.

``They followed the stones to the top of the hill to watch their
proceedings there. The stones in several places were seen huddling
themselves close together, and there were some others rolled up to them,
and gave one hop, and jumped on to the top of them, till at length there
were seen piled up in just such heaps as they lay in before down below.
Then the pegs --- the architect's pegs were hopping about upon the
ground like sparrows; but their wooden heads did not seem to be half so
sharp-witted as the stones, for they seemed sorely puzzled where to
place themselves, notwithstanding the apparent exertions of a tall wand,
with a bit of coloured rag at the top of it, which kept constantly
moving backward and forward, now sticking himself in at one corner, and
then at another, probably much in the same way that it had previously
done under the architect's directions. But long before they had made
their arrangements to anything like their own satisfaction, up hopped a
spade, which banged across the ground they were marking, knocking down
two or three pegs in his way without any ceremony, and began sedulously
digging and throwing out the earth. It was marvellous to see how it
crammed itself into the ground, and then threw out the earth, without
any hand or foot to guide it.

``When it had dug a hole sufficiently large, up rolled a large flat
stone, and squatted itself down in it. The stone was afterwards found to
be the same identical stone that had been laid by the bishop with so
much ceremony down below.

``This was the account given by the two men who had been set to watch.

``One of the other men employed now stepped forward, and said, that with
regard to the bad hurt that one of the watchers had got upon his shin,
he was quite certain that his companion had not received that hurt up to
late in the evening before. They always worked with bare legs, and he
must therefore have seen it.

``Here the bishop and his council put their heads together, and
consulted a little in an under tone. It was evident that the man had
received his hurt some time during the night, and not during his work
hours; and it was quite incredible that he could purposely have
inflicted such an injury upon himself. This was a strong piece of
circumstantial evidence, and went far to prove the truth of the story.
Then the account given by these two men agreed so exactly in every
particular, --- they were so accurate in the description of every minute
circumstance, --- all the different parts of the story fitted so well
together, that they considered it unnecessary to hear any farther
evidence upon the subject The bishop then dismissed the assembly.

``Two days after this the bishop, attended by the principal
ecclesiastics and the chief inhabitants of the Isle of Wight, went in
solemn procession to consecrate the new site of the church. The ceremony
was very similar to the preceding one, except that the bishop recited at
great length, and with some trifling alterations and additions, the
wonderful miracle that had taken place upon the spot. After he had
concluded his address, they raised the foundation-stone to see whether
the piece of money was still lying under it.

``Great was the astonishment of all the assembly to find that it was
gone, and exactly in the spot where it should have been was found the
paring of a thumb-nail As soon as this was publicly announced, a loud
and universal shout arose --- `A relic! a holy relic!' I pass over
altogether, for it would be grating to the ear of every religious
Protestant, the consultations that were held upon the subject, the
processions that followed, the masses that were said, the adorations
that were paid to this trumpery and filthy object. It is sufficient to
know that the site was consecrated, the church was built, and the ground
upon which it was erected has ever since been known by the name of
God's-hill.

``The franklin was highly pleased to have had all his sins absolved by
the bishop himself, without the necessity of any confession; while his
cows still ranged over his favourite field; and the two watchers never
passed that way without partaking of the best cheer that the franklin
could set before them.''

\bookmarksetup{startatroot}

\chapter{Godshill Tall Tale}\label{godshill-tall-tale}

\emph{If you're ever minded to descend into the archives in search of
Isle of Wight folklore, you'll almost certainly meet Abraham Elder's
``Tales and Legends of the Isle of Wight'' along the way. This early
19th century roadtrip featuring the author and his antiquarian sidekick,
Mr Winterblossom, sees the two travellers driving around the island, in
a horse and trap presumably --- this was in the days before steam
railways, let alone motor cars --- in search of local stories. But many
of the tales are relocated tales from elsewhere, and some have the
feeling that they are just plain made up. And even the author's name,
Abraham Elder, is a pseudonym: the best guess is the author was Augustus
Moreton, MP for Gloucestershire, who stayed regularly with his uncle, in
Bembridge. Anyway, this tale is one of Elder's, taken from Bentley's
Miscellany in 1839 or so, a literary journal edited around that time by
a certain Mr Charles Dickens. It's a version of a legend that you may be
familiar with: how Godshill Church came to be situated just where it can
still be found today.}

The tale begins with the raising of funds to build a new Church, and a
search for where to situate it. The perfect spot is found, a flat plain,
near a hill, between Whitwell and Arreton, between Rookley and Wroxall,
and the Bishop approves. It was a very fine site indeed.

The man who farms the land, a poor franklin, or freeholder, is informed
by a monk from the Abbey of the great honour to be bestowed upon him.
But rather than praising the Lord, the man protests: ``I'm not a pious
man, sir'' he says, ``it would not do to put the Church there, and what
would I do, it's my best bit of land'', and on and on he goes. And it's
true. He's not a pious man at all. Indeed, he'd rather go out of his way
to avoid a confession, rather than to make one, and on the occasions
when a certain passing friar would take his confession, well, it tended
to be after supper, and after a pint or two. And if penance were called
for, well then, the ale might turn a little sour, or be offered no more.

``My neighbours, sir, my neighbours. They are both men of God, sir, both
men of God, good men, not a sinner like me. And their land is surely
holier than mine, sir. And closer to god, closer to God'' , and he
pointed up the hill to his neighbour's land.

But the monk \emph{wouldn't} be persuaded, indeed, he \emph{couldn't} be
persuaded, because the choice had been made: ``your unholy nature is no
matter'', he said, ``when the foundation stone is laid, you will be
absolved of your sins''. And even as the man protested that his
neighbours were more pious than he, more deserving, the monk walked away
without even a promise of compensation.

The stones to begin the construction duly arrived from a local quarry,
the site was marked out with pegs and ropes to the satisfaction of the
architect, and the day to consecrate the land and lay the foundation
stone duly arrived. The architect and the mason arrived good and early,
and\ldots{} where were the stones? Where were the pegs, and the ropes?
They looked around. Surely they couldn't be in the wrong field? They
called out to Master Franklin, who's head could be seen bobbing up and
down behind a nearby hedge: ``what tricks have you been playing, sir?''.
But he stood there, as bold as you liked, gesturing to himself, and
shrugging, as if to ask: what, me? What do you mean?

Just as the men really were starting to doubt that they had right field,
the owner of the land at the top of the hill turned up, and he was in
angry mood, a gravely offended mood. ``What in the Devil's name are you
doing?'' he said. ``I've heard tell that the Church will take a man's
land for no compensation to put a Church on it, but to start the work
without even telling the man, that's not right. That's not right at
all'', and on and on he went.

Now, the architect had no idea what the Franklin's neighbour was talking
about. ``What am I talking about? What am I talking about? That's what
I'm talking about\ldots{}'', and he pointed up the hill; and then he
started to drag them up the slope, ``come on, come on''\ldots{} Well,
they hadn't gone far when they saw the stone, and the pegs, and the
ropes, laid out exactly as they had been before; only this time, on top
of the hill.

Well, they immediately set to, and with the other masons and builders
who had by now started to arrive and follow the commotion, they took
everything back down the hill, and put it back how it was. Just in time
for the Bishop and his retinue to arrive, the Bishop, seated, royally,
looking as princely as he could, given that he was sitting on a donkey.

But rather than stopping at the appointed place, the donkey resolutely
walked straight past the building site, indeed, it quickened its pace,
if anything, and started to make its way up the hill. The assembled
monks, who had been chanting psalms and looking heavenwards to prepare
that place for the solemn ritual about to take place, broke off and
chased after the Bishop and his wayward donkey. Eventually they caught
up with him, half way up the hill, and led them both back.

The Bishop was by now not in the best of moods, but he was there to do
God's work, and do it he would, and so he got off the donkey, and laid
the foundation stone, placing under it a coin and some masonic tools.
And then he absolved all the sins of the franklin in exchange for the
land and set off back for the Abbey.

As the party headed away, the conversation was dominated by the strange
story of how the stones had somehow moved to the top of hill the
previous night. If they'd been placed on top of the hill and rolled
down, well, that would have been weird, but it sort of makes sense.
Things are always rolling down hills, but not usually uphill.

Or perhaps it was a miracle. Now, a certain sort of a churchman really
likes the idea of a miracle, and the pilgrim benefits that accrue, but
the majority thought it unlikely to be a miracle. There was surely
another explanation?

Well, then, a joke, perhaps, a prank? But who would do such a thing? Not
the pious men of the island, not men who might face excommunication and
eternal damnation for thwarting the will of God and the placing of his
Church.

So talk turned again to the possibility of the miracle, and one of the
party feverishly pointed out how the donkey had also made its way past
the original site, and up the hill, and how that too was surely a sign,
was surely more evidence that a miracle had occurred?

Now, the Bishop himself wasn't convinced by this. Not least because the
Bishop and the donkey had previous form. And so the Bishop suggested
that one of the monks go back and tell two of the workmen that they
should keep watch over the stones that night, and then after they
reported back in the morning, they could discuss the matter further. And
so it was done.

The next day, a messenger arrived at the Abbey with the strangest of
news: he'd found the two watchmen lying helpless on the ground at the
foot of the hill, with sore heads and bloodshot eyes, and they had told
the most marvellous tale; and the stones, the stones that had been at
the foot of the hill, including the foundation stone, had all
miraculously been transported back up the hill again. And the site at
the bottom of the hill was all repaired so as you wouldn't know that the
foundation stone had been laid there the day before at all.

Well, the two watchmen were summoned, and asked to explain themselves
and they told of how at first the little stones, had start to jiggle,
and almost jump, and then start to roll and bounce their way up the
hill. And one of the stones had bumped and jumped and clattered straight
into tone of the men's shins --- and at this point, he showed a fresh
bruise on his leg. And then the pegs and the ropes rolled themselves up,
and up the hill, and then the bigger stones, well, it was as if they
needed a few attempts to lift themselves and start rolling but they did
it. And one stone, well, it had an odd shape, with a jagged bit, and as
it rolled it kept getting stuck. And it was so sad to see it struggling,
that the men thought, should they show it some charity and help it out.
And so they did, they gave it a hefty push to help get it out of a rut
and then it seemed to gain momentum and it rolled away and it managed to
make its way the rest of the way up the hill, all by itself. And as the
Bishop interrogated the men, well: their stories barely had a hair's
breadth of a difference between them.

The Bishop looked on in wonder, and called for the whole party to return
to the site so he could see this for himself. And so the donkey was
saddled, and they set off once again. Getting to the top of the hill,
the pegs marked out the Church as exactly as they'd been placed before,
and even the foundation stone was where you'd expect. At least, if you
expected to find it marking out a church built on top of the hill.

The final proof would surely be if the coin and the mason's tools were
to be found under the foundation stone. And so it was lifted, and there,
there beneath the stone, was: a miracle. A relic. A holy relic. A
thumbnail. And the monks began to chant and sing, and praise the holy
lord and the masses that were said that day, and the processions they
made, well, you'd think it was a relic of the Holy Lord Jesus himself.
And perhaps it was.

By now, it was obvious that God's work was afoot, and that the top of
the hill was surely the right place for the Church. And so another
consecration ceremony was held, but this time on the top of the hill.
And there the Church was built; and there it still stands today.

And at the bottom of the hill, what of the franklin who lived there?
Well, his mortal sins had been absolved the first time around, and he
continued to work that land till the day he died. And if either of the
two watchmen from that earlier night were to ever pass by his way, well,
he'd invite them in and they'd share an ale or two and they'd retell the
tale of what happened on that miraculous night, just as I've told you,
and as you can tell others.

And that is the end of the story.

\bookmarksetup{startatroot}

\chapter{\texorpdfstring{In Percy Stone's \emph{Legends and Lays of the
Wight}}{In Percy Stone's Legends and Lays of the Wight}}\label{in-percy-stones-legends-and-lays-of-the-wight}

In \emph{Legends and Lays of the Wight} first published in 1912, local
architect, author and archaeologist, Percy G. (Goddard) Stone quotes the
summary of the legend of Godshill from \emph{Barber's Picturesque
Illustrations of the Isle of Wight}, before retelling the tale more
colourfully in verse:

\begin{quote}
They had gathered them in from the countryside round\\
To settle a serious matter.\\
The question was this:\\
They sore needed a church\\
But they seemed very like to be left in the lurch,\\
And the whole thing to finish in chatter.

Shall we build it up there on the top of the down,\\
Or here in the valley below?\\
As at all parish meetings, a number said Aye\\
The hill is the spot for it's open and high,\\
While the rest called them fools, and said No.~
\end{quote}

\begin{quote}
And while they thus argued for valley or hill\\
In the weary parochial way.\\
The Devil drew near, in the guise of a Friar\\
--- The Devil is always a plausible liar,\\
And put in his word for the Nay.

`Why burden yourself with the onerous task\\
Of building up there on the height?\\
The carriage of stone will entail a sad toil,\\
So listen to me, honest sons of the soil.\\
Below's the most practical site.'

`The holy monk's right,' shouted those who agreed\\
To the vale scheme. Beelzebub lied.\\
As he always can do, with such hearty good will\\
That most of the folk who'd declared for the hill\\
Turned about and came over his side.

Satan reckoned, if tucked away snug out of sight,\\
'Twould be more out of mind of the Saints.\\
Besides, with the muddy and waltorish ways,\\
They would sure be more likely to curse than to praise,\\
Thus their prayers would get choked with complaints.

St.~Boniface, passing, soon saw what was on,\\
So put in his vote for the hill.\\
Don't listen to fossils like him~---\\
Slaves to orthodox precedent, crotchet and whim,\\
Quoth Satan, `Tis clear, friends, his knowledge is nil.'

So to it they went with a hearty good will.\\
Said the Saint, `It is just as I feared:\\
Though the fools wouldn't listen to what I'd to say,\\
Beelzebub shan't have it all his own way'\\
And he thoughtfully pulled at his beard.
\end{quote}

\begin{quote}
So the masons hewed stones and then set them in place,\\
Chipping hard till the daylight had sped.\\
Then gathering their tools, with a sigh of content\\
At the work they had done, gaily homewards they went\\
To supper, and after to bed.
\end{quote}

\begin{quote}
Next morning betimes they arose with the lark\\
And set forth to work with a will.\\
But lo! when they got there, 'twas level with ground.\\
--- Not a sign of a stone to be anywhere found\\
They were all laid atop of the hill.

`Here, bother it all!' the head mason exclaimed\\
`Who's been playing the fool with this job?'\\
`If I catch him, mark well' --- and he doubled his fist,\\
With a look that I wouldn't for money have missed,\\
``I'll trounce him right soundly begob!''

His comrades, sore puzzled, looked up and looked down,\\
Then exclaimed with unanimous voice\\
--- They knew nothing about it so strike 'em all dead\\
As after their suppers they went straight to bed,\\
And had naturally stayed there, for choice.

So it's up to the top of the hill they must toil\\
And fetch all those stones down again.\\
While they swore at the fool who had played them the trick\\
--- Of course the first person they thought of was Nick,\\
And given them toiling in vain.

Once again in their places the pieces secured,\\
They mortared them surely and true.\\
But when morning dawn lightened, the stones as before\\
--- Whereat the whole company lustily swore\\
Were ranged on the hill-top anew.

Said the stout master mason, a-scratching his head,\\
`This is getting too much of a jest;\\
Let alone double toil, it will never get done:\\
'Tis the job of a lifetime for every one:\\
From our labour we never shall rest.'

When the very next morning the same thing occurred.\\
`Stop, mates: let it be. The whole thing's\\
A puzzle too tough for my nob;\\
The Building Committee must settle this job:\\
It's a bit too perplexing for me.'

So they called them together and met once again\\
And argued it worse than before.\\
They argued it up and they argued it down,\\
Like the council elect of a small county town,\\
Till they argued their very throats sore.

Then Boniface Saint he laughed low in his beard:\\
``'Tis Isle of Wight calves that you be.\\
Are your heads then so thick that you can't understand\\
When from Heaven you're graciously sent a command,\\
It 's a miracle, easy to see.''

`You scorned the advice of a credited saint'\\
--- Here his stature a full cubit rose,\\
`Who was never convicted of being a liar,\\
And hung on the words of this sham shaven friar,\\
Who led all you fools by the nose.'

``A friar forsooth! 'Neath his cowl and his frock\\
Horns and hooves I can clearly perceive.\\
--- This has been one of your narrowest shaves ±\\
Come hither, Beelzebub, subtlest of knaves,\\
And your recompense duly receive.''

Saint Boniface here gripped him tight by the neck,\\
``With, None of your sly monkey tricks''\\
Then, catching him fair with the point of his toe,\\
He lifted the Devil a furlong or so\\
With a couple of right lusty kicks.
\end{quote}

\begin{quote}
Whereat the parishioners fell on their knees\\
And lauded the Saint to the sky.\\
So Boniface blessed them in orthodox way.\\
Then, looking around him, proceeded to say,\\
'This humility's mostly my eye.

'Next time don't be led by a plausible rogue\\
Unknown to the Parish and you\\
Though I've heard it, my friends, reprehensibly said\\
By Christians, alas! who are easily led,\\
We must render the Devil his due.

`Avoid mendicant friars, and take this advice\\
I tender you ere I depart:\\
Don't argue it more. Work away with a will\\
At building your church on the top of the hill,\\
Every man of you bearing his part.'

`We assure you, good Saint, we will do as you say.\\
But, ere that you bid us \emph{adieu},\\
Grant this favour benign, that our newly-built church\\
--- By Beelzebub's guile nearly left in the lurch,\\
May be titled in honour of you.'

``Though I'm flattered, the honour I needs must decline\\
--- Besides I'm bespoken elsewhere\\
When everything's finished, to carving and paint,\\
Let your church be invoked in the name of each saint\\
By whom you're accustomed to swear.''
\end{quote}

\begin{quote}
All humbly they cried, `It shall be as you wish.\\
Your Saintship; we'll work with a will\\
And the site where the building in future will stand,\\
According to God's and your saintly command,\\
Shall be known by the name of GODSHILL.'
\end{quote}

\begin{quote}
The church there to-day on the top of the mound\\
Lifts its pinnacled Tower to sky\\
While a mile to the South Devil's Acre is found\\
--- Which maids in the dark are afraid to pass round,\\
Two proofs to you Legends don't lie.
\end{quote}

\emph{\texttt{Isle\ of\ Wight\ calves}: a local expression for a
dull-witted man.}

\bookmarksetup{startatroot}

\chapter{Nineteenth Century Tourist Guide
Descriptions}\label{nineteenth-century-tourist-guide-descriptions}

Since the late seventeenth century, tourist guides and itineraries have
featured the Isle of Wight as a well-regarded destination. Many of these
guides provided descriptions of the island towns and villages as they
were at the time, as well as commenting on notable attractions and
scenery, and, on occasion, local legends.

Black's \emph{Picturesque Guide to The Isle of Wight} from 1876 paints
the following picture of Godshill:

\begin{quote}
This, the ``most romancy'' (as old Aubrey would say) of the
island-villages, abounds in bloom and leafiness, out of whose balmy
depths rises the rugged church-crested hill, its abrupt sides studded
with irregular cottages, and broken into flowery rifts and chasms. The
Church, dedicated to All Saints, is worth a visit, as well on account of
its architectural merits and interesting memorials, as of its admirable
and striking position. A panorama, only to be described by a poet,
greets the spectator's eye from this insulated point. ``To the north the
gaze embraces the whole of the vale of Newchurch, with the undulating
ridge of the chalk downs beyond, ending towards the valley of the Medina
in the abrupt slope of St.~George's Down. The white cliff of Culver are
just descried over some rising ground to the right; to the left we have
the ridge separating the valleys of the Yar and Medina, and the bold
line of chalk downs which here take a due southerly direction. To the
south the view is more varied. The northern front of the southern chalk
range, with its bold projecting spurs, and sinuous valleys lies before
us. Appuldurcombe, or Week Down, with its shattered obelisk, bold wall
of cliff (the northern face of the firestone stratum, which gives its
picturesque character to the Undercliff), and rich hanging woods, rising
immediately in front over the scattered houses and leafy knolls of the
village; to the west is the huge mass of St Catherine's, marked by the
twin pharoses, and the slender Alexandrian pillar; to the east rises the
more picturesque outline of Shanklin Down, with its belt of timber half
concealing its cliffs, on the summit of which stands the modern ruin of
Cook's Castle'' --- \emph{(Venables).}
\end{quote}

A brief history of the church is also provided, including a reference to
a famous lightning strike and a mention of the legend regarding the
building of the Church:

\begin{quote}
Godshill was one of the six churches with which William Fitz-Osbert,
after the Norman Conquest, endowed his favourite abbey of Lire, in
Normandy. Charles I presented it to Queen's College, Oxford. It was much
injured by lightning in January 1778. A tradition (of no uncommon
character) attempts to account for the peculiar name of the village. Its
builders first proposed to erect it at the foot of the hill, but every
morning found the preceding day's work undone, and their material
carried to the summit. After a few days' perseverance they wisely
resolved to struggle no longer against the invisible workmen, and built
the church on the site indicated by the spirits, where it still stands
--- to all the country side around a stately beacon of the Christian
faith.
\end{quote}

Barber's \emph{Picturesque Illustrations of the Isle of Wight},
published in 1850, describes Godshill a little more succinctly as
\emph{``a picturesque village, much remarked for the bold position of
its Church, which, standing on a very abrupt hill, surrounded by the
cottages of the inhabitants, may be supposed to have given name to the
place, from its erection on so commanding a site''}. The legend is then
described as a \emph{{[}r{]}ustic tradition} that related how \emph{``a
more lowly spot was first selected for the erection of the church; but
that the materials employed for that purpose, day by day, being
regularly removed by invisible agents to the summit of the hill during
the night, the workmen at length wisely determined to save themselves
further unnecessary trouble, and built the church where some
supernatural authority so plainly intimated that it must be erected.}

Cox's \emph{Isle of Wight: its churches and religious houses} published
in 1911, gives the legend as follows:

\begin{quote}
An old legend, still current in the parish and district, tells how the
inhabitants, when first Christian preaching had won them to the true
faith, began to build a church on a level place a mile to the south-west
of the present village, but as fast as they laboured during the day,
when nightfall came the stones mysteriously disappeared, and were at
last found on the top of the hill. Recognising that this must be the
will of God, they decided that the church was to be completed on this
lofty site, and from that time both the church and the village, which
grew up around it, have been known by the name of Godshill.
\end{quote}

An even more religious telling is provided in ``The Isle of Wight'' by
George Clinch, published in 1908, quoting an \emph{``ancient legend
which is thus given in a local guide-book''}:

\begin{quote}
The people of this village, having long lived in pagan darkness, were at
length visited by a holy man who came and lived among them. He told the
wondrous story of Divine self-sacrifice, and taught the Catholic
religion of love and mercy. This so touched the hearts of the people
that they cast down their blood-stained altars and acknowledged the true
God.

Then cried the elders of the village: ``We will build a temple to His
honour, where we and our children's children may worship, and by which
generations yet unborn may know how the Saxon reverenced God.''

They chose, therefore, a level place, and all day long they wrought and
toiled, and when night came they rested from their labours. But on the
morrow, when borne during the night by invisible hands to the top of a
round knoll.

When the people saw this marvellous sight they cried with one voice:
``Let us build now on the top of the hill, for this must be the will of
God,'' and from that time the church and village have been known as
Godshill in memory of this great deed.
\end{quote}

Clinch goes on to describe how \emph{{[}t{]}his legend, which may have
been invented in explanation of a rather remarkable name, belongs to a
class of which numerous examples still linger in England''}:

\begin{quote}
Usually one finds them in places where the parish church stands at a
considerable distance from the village, and, generally, they are to the
effect that the building erected in or near the village during the
day-time was conveyed by evil influences to a remote situation at night.
The Godshill legend is a variant of this group.
\end{quote}

\let\thefootnote\relax\footnotetext{\textit{Original copies of the above works can be found in the Castle Museum library at Carisbrooke Castle. The collection is viewable by appointment.}}

\bookmarksetup{startatroot}

\chapter{Abraham Elder's Tales and Legends of the Isle of
Wight}\label{abraham-elders-tales-and-legends-of-the-isle-of-wight}

One of the few purported 19th century collections of Isle of Wight
folk-tales is Abraham Elder's \emph{Tales and Legends of the Isle of
Wight}, first published in 1839. The tales are presented as if in a
dialogue between the author, and his traveling companion around the
island, a certain Mr Winterblossom.

According to a correspondent in the Isle of Wight County Press, a
newspaper that continues to be published to this day, of Saturday 16
December 1893,
\href{https://www.britishnewspaperarchive.co.uk/viewer/bl/0001960/18931216/025/0002}{p2},
\emph{Abraham Elder produced in that year {[}1839{]} two issues of his
``Tales and Legends.''}

\begin{quote}
The first issue was a small book of 196 pages with only four
illustrations, drawn, I conjecture, by Abraham Elder himself. The
illustration for page 4 has the curious blunder of the number 116 being
substituted for 4. This smaller issue is called on the title page ``Part
the first.'' This issue has a value of its own, as it contains as a
frontispiece a view of the Needles, showing the original needle rock.
This frontispiece may have been meant by Abraham Elder to be included
amongst the legends of the Isle of Wight. In the same year came out the
second issue of the ``Tales and Legends.'' This issue is a much larger
volume, it contains 336 pages, with 14 large and 11 small woodcuts by
Robert Cruikshank. The illustration of page 4 of the first issue is
reproduced as a vignette on page 8 of this second issue, otherwise the
25 illustrations are entirely different from the illustrations of the
first issue. Of this second issue a second edition appeared in 1843, but
in no way differing from first edition. \ldots{} R. G. D.
\end{quote}

Elder used the preface of the book to provide some background, or so we
are led to believe, for the provenance of the stories in the collection.
In so doing, he also reveals that he is not an Islander himself:

\begin{quote}
It may appear singular that the present little volume, intended to
illustrate the antiquities and traditions of the people of the Isle of
Wight, should come from the pen of a stranger; and that the tales it
contains should have been collected during the space of a short summer
ramble. Fortune, however, has favoured me much, and I take this
opportunity of returning my thanks to my friend, the well-known and
justly celebrated antiquary, Mr.~Winterblossom, for his most valuable
contributions, without which this collection of Tales and Legends would
have been but chaff.
\end{quote}

Rather unusually, Mr Winterblossom's credentials are proved by what he
has \emph{not} achieved, specifically, publishing a collection of tales
himself:

\begin{quote}
I must confess that I am rather a collector, compiler, or editor; though
vanity (as some would term it) has induced me to call myself the author
of the present work. I am but the wisp of straw that ties the fagot
together. But it is not altogether vanity in me either: for I have
repeatedly, though in vain, urged Mr.~Winterblossom to undertake the
task. But he said that he had once spent many years and infinite labour
in preparing a work of deep erudition for the press. The public was
ungrateful, and the work still remains unsold. Printer's ink he would
never meddle with again; and then, added he with emphasis, taking me by
the arm, ``the printer will be paid, whether the work sells or not''.
\end{quote}

We also get a hint from the preface that if the current work meets with
some success, Elder would happily produce a second volume:

\begin{quote}
Should, however, the public act with greater indulgence in the present
instance, it is my intention, if life and health be preserved to me, to
offer a second volume to its notice in the due course of time. One
difficulty presents itself.
\end{quote}

To compile such a work, however, Elder suggests he would benefit from
local assistance in sourcing or uncovering local tales. Notably, he
gives a local bookseller as one of his contact addresses.

\begin{quote}
Being a stranger to the island, --- during the short time I may be able
to bestow upon a visit next summer, I may perhaps overlook many things
that ought to be recorded with care. I should therefore feel under
obligation to any one who will call my attention to any antiquity, tale,
or tradition connected with the island; --- directing their
communication to A. Elder, Esq., author of '' Tales and Legends of the
Isle of Wight,'' under cover to his publishers, Messrs. Simpkin,
Marshall, and Co., or to Mr.~French, bookseller, Newport.
\end{quote}

Although a second \emph{volume} of tales did not appear, a second
\emph{edition} of \emph{Tales and Legends} was published in 1843. This
edition contained several additional tales that had also appeared
throughout 1839 and 1840, in the literary magazine \emph{Bentley's
Miscellany}. The first editor of \emph{Bentley's Miscellany}, which was
first published in 1836, is perhaps better known today as an author in
his own right: a certain Mr.~Charles Dickens, whose book \emph{Oliver
Twist} was originally serialised in that magazine. Dickens, who was born
in Portsmouth, is known to have visited the Island in September, 1838,
and returned again for a longer stay in the summer of 1849. It is
interesting to speculate whether he had met up with one of his
contributing authors on that earlier visit. As the following letter from
volume one of \emph{The Pilgrim Edition} of \emph{The Letters of Charles
Dickens} shows, Dickens' mind was at least partly on the job even if he
was on holiday, although he was, perhaps, looking forward to getting
home?!

\begin{quote}
To Richard Bentley, {[}10{]} September {[}1838{]} \emph{Ventnor, Isle of
Wight. \textbar{} Monday Evening, Septr. 9th} My Dear Sir, I have
received a letter with some double or treble postage tacked to it from
one ``Omega'' \emph{{[}an unidentified contributor whose to the
Miscellany had been declined in Notices to Correspondents in April,
1838{]}} relative to some german translations. This Omega is an
impertinent gentleman who has done me the favour to write a continuation
of Pickwick and ask me to father it. In case you should hear from him
again, I think it best to let you know that I have written him a reply,
stating that I do not wish to accept his papers, firstly because the
greater part of them are too long, and secondly because you do not
require translations from new Contributors --- the strong probability
being (this I have \emph{not} told him) that they have appeared
somewhere else before.

I shall be home, please God, on Wednesday Evening. Shall I see you this
week?

Faithfully Yours

Charles Dickens
\end{quote}

The tale of the Godshill legend, included in this volume, was originally
published in \emph{Bentley's Miscellany}.

\section{Who Was Abraham Elder?}\label{who-was-abraham-elder}

In 1873, scholarly story collector Joseph Jacobs' published \emph{More
English Fairy Tales}, his second collection of English fairy tales. The
volume included a version of Elder's legend relating to the Pied Piper
of Newtown, a story that appears to have made its first appearance in
\emph{Tales and Legends of the Isle of Wight} in 1839.

Whether the legend was a ``true'' one is a story for another day, but
the publication of tale brought some attention as to who the real author
might actually be, Abraham Elder having been identified as a \emph{nom
de plume}.

In the August 30th, 1873, issue of \emph{Notes \& Queries},
\href{https://archive.org/details/sim_notes-and-queries_1873-08-30_12_296/page/168/mode/2up}{Vol
12 Iss 296}, a 19th century periodical, published weekly since November
3rd, 1849, which acted as a \emph{``Medium of Intercommunication for
Literary Men, Artists, Antiquaries, Heralds''}, the following query
appeared:

\begin{quote}
``TALES AND LEGENDS OF THE ISLE OF WIGHT: with the Adventures of the
Author in search of Them.'' By Abraham Elder, Esq. 2nd edition, 1843.---
Who wrote this work? It is not mentioned in Mr.~Olphar Hamst's
\emph{Handbook}. Mr.~Abraham Elder was evidently a person of culture and
research, possessed of a delicate humour and much literary skill. His
book is very interesting, and might well be reprinted. \ldots{} A. J.
Munby.\\
Temple.
\end{quote}

It seems the question went unanswered.

The same question was asked by the Isle of Wight County Press
correspondent \emph{R. G. D.} in 1893:

\begin{quote}
I have tried to find out the real name of Abraham Elder. I have learned
that he either was native of, or lived in, Shanklin and that to his
parents he owed the name of Clayton, but I have not yet been able to
find out what name was given him by his godfather and godmother. I
should be much obliged to any of yours readers who would give me this
information.
\end{quote}

In reply was a letter that just missed making it into the next week's
edition who is makes some uncertain suggestions as to the identity of
the mysterious author:

\begin{quote}
ABRAHAM ELDER

\emph{To the Editor of the Isle of Wight County Press.}

Sir,--- Abraham Elder's ``Tales and Legends of the Isle of Wight''
contains a great deal of matter which has but little relation to, and
which is by no means the natural product of, the Island, but which
originated only in the fertile imagination of A. E., whose \emph{real}
name is shrouded in mystery, like the name of the builder of the
pyramids. For years I have vainly endeavoured to discover it and still
like Junius, \emph{stat nominis umbra}. I do not think he was a native
of the Island, but probably a visitor, or some one who had been resident
for some time. I have heard the names of Clayton, Moreton, and others
mentioned in connection with the tales, but nothing of satisfactory
nature giving any real clue to the writer's identity.
\end{quote}

The correspondent also reviews the publication history of Elder's work
and passes comment on what they know of contributors to Bentley's
magazine:

\begin{quote}
The second part of the ``Tales and Legends'' first appeared in the pages
of \emph{Bentley's Miscellany} in 1839 and 1840. I knew one of the
earliest contributors to \emph{Bentley}, and the last survivor of that
brilliant circle of writers, a gentleman who was well acquainted with
most of the contributors to the earlier volumes of the
\emph{Miscellany}, and through him I made inquiries few years ago of
Bentley and Son to the real designation of A. Elder, but could obtain no
reply, after much research. His identity was forgotten and remained
unknown to his publishers. The first part of the tales, as your
correspondent ``R.G.D.'' observes, was published in 1839, with four
plates in lithograph.
\end{quote}

The correspondent also seems to be of a different opinion to R.G.D. as
to when the second issue of \emph{Tales and Legends} was first
published:

\begin{quote}
Another issue was published in 1841, with plates by R. Cruickshank,
engraved by the ``gypsographic'' process. In this volume the second part
of the tales was first published as a book, and with the first part
makes a volume in 12mo of 336 pages. A second edition appeared in 1843,
and is simply reprint of the volume published in 1841. All the editions
are now become scarce, the first especially.--- I am, yours truly,

W. H. Long. 120, High-street, Portsmouth, December 22, 1893.
\end{quote}

However, a response that did make it in time to be included in the
\href{https://www.britishnewspaperarchive.co.uk/viewer/bl/0001960/18931223/040/0003}{Saturday
23 December 1893} edition of the \emph{Isle of Wight County Press}, p3,
was more certain of Abraham Elder's identity:

\begin{quote}
ABRAHAM ELDER.

To the Editor of the Isle of Wight County Press. Sir,

In reply to the inquiry of your correspondent R.G.D., I can state that
author of the ``Tales and Legends of the Isle of Wight'' was the late
Hon.~Augustus Moreton, M.P. for Gloucestershire, who wrote the book
under the name of Abraham Elder when living with his uncle, Col.
Moreton, about that time at Bembridge. ---Yours truly. W.W. Osbourne,
Bembridge, IW., Dec.~21. 1893
\end{quote}

It seems the Editor of the Isle of Wight County Press was happy to
assertion, as we can tell from an Editor's comment added to Mr Long's
delayed reply in the edition of a week later:

\begin{quote}
{[}Mr.~Long's letter reached us too late for insertion in our last
issue, in which, it will be remembered, it was asserted on the authority
of Mr.~W. W. Osborne that the author of the book in question was the
Hon.~A. Moreton.---ED. \emph{I.W.C.P.}{]}
\end{quote}

Born in June, 1804, the Hon.~Augustus Henry Moreton Macdonald of Largie
(born Augustus Moreton) was the younger brother of Henry
Reynolds-Moreton, 2nd Earl of Ducie, and son of Thomas Reynolds-Moreton,
1st Earl of Ducie, and Lady Frances, daughter of Henry Herbert, 1st Earl
of Carnarvon. He was elected as a Member of Parliament for
Gloucestershire West in 1832 and then for Gloucestershire East between
1835 and 1841. A campaigner for homeopathy, he had published various
works under his own name, including \emph{Civilisation, or, a Brief
Analysis of the Natural Laws that Regulate the Numbers and Condition of
Mankind} in 1836 and \emph{Thoughts on the Corn laws, addressed to the
working classes of the county of Gloucester} in 1839.

If we accept W.W. Osbourne's suggestion that Abraham Elder was, in fact,
the Hon.~Augustus Moreton, M.P, a claim mildly supported by W. H. Long's
suggestion of possible names, then what evidence can we find of the
honourable gentlemen on the island?

From a brief mention in
\href{https://www.british-history.ac.uk/vch/hants/vol5/pp156-170}{\emph{A
History of the County of Hampshire}: Volume 5, ed.~William Page (London,
1912}, we can probably vouch that his uncle did indeed live on the
Island:

\begin{quote}
Bembridge.

In 1854 Colonel the Hon.~Augustus John Francis Moreton by his will,
proved 5 September, left £300, the interest to be given to deserving
poor. The legacy was invested in £327 5s. 8d. consols, producing £8 3s.
8d. yearly.
\end{quote}

We also see from a news report in the \emph{Hampshire Telegraph} of
\href{https://www.britishnewspaperarchive.co.uk/viewer/bl/0000069/18351019/006/0002}{Monday
19 October 1835}, p2, that uncle and nephew were active in local good
works:

\begin{quote}
Public Meeting of the Inhabitants of Bembridge, held at the Free School,
on Monday the 12th of October, 1835.

Honourable A. MORETON, M.P. Chairman

Proposed by Mr.~John DENNET, and seconded by John Newman: 1. That this
meeting deeply sensible of the important and valuable assistance
rendered to the Education of the Children of the Poor by the Rev.~Sir
Henry Thompson, Bart. in not only erecting the School House at his own
expense, and paying the salaries of the School Master and Mistress, but
also for his unwearied personal attention to the moral and religious
instructions of the Inhabitants generally, do hereby tender him, their
grateful thanks, and express their grateful and sincerest wishes for his
future welfare and happiness.

\ldots{}

\begin{enumerate}
\def\labelenumi{\arabic{enumi}.}
\setcounter{enumi}{2}
\tightlist
\item
  That a Committee be formed, consisting of: The Hon.~Colonel Moreton,
  the Hon.~Augustus Moreton, \ldots{}
\end{enumerate}

\ldots{}
\end{quote}

We also have evidence of Moreton's social standing on the Island, as
noted in this report from the \emph{Hampshire Advertiser} of
\href{https://www.britishnewspaperarchive.co.uk/viewer/bl/0000494/18450607/034/0008}{Saturday
07 June 1845}, p8:

\begin{quote}
Cowes, Saturday June 7

Royal Yacht Squadron Intelligence\\
Arrivals

June 3 \emph{Elizabeth} Hon Augustus Moreton, from Guernsey

Fashionable Arrivals. The following are among members of the R. Y. S.
who have visited the Squadron House during the week: \ldots,
Hon.~Augustus Moreton, \ldots{}
\end{quote}

\bookmarksetup{startatroot}

\chapter{Afterword}\label{afterword}

If you do make it up to the hill to Godshill Church, and browse the
noticeboard there, you're as likely as not to see a poster for a flower
festival, or a jumble sale, perhaps, as well as sundry other church
notices.

But for one visitor, John Hassell, writing in the second volume of his
two volume work, \emph{Tour of the Isle of Wight}, published in 1790,
and which you can find nowadays in the Castle Museum library (\emph{by
appointment}) at Carisbrooke Castle, or via the
\href{https://archive.org/details/bim_eighteenth-century_tour-of-the-isle-of-wigh_hassell-j_1790_2/page/n121/mode/2up?}{Internet
Archive}, there was something rather more unusual posted there. After
noting damage to two of the gable ends resulting from the 1178 lightning
strike, Hassell recounts his experience of entering the Church (p.88):
\emph{{[}u{]}pon our entering the porch we observed abstracts from
several acts of parliament fixed against the door, and among them one
that excited both our curiosity and risibility;}

\begin{quote}
--- it was from an act made in the seventh of James the First, which
enacts, that every female who unfortunately intrudes on the parish a
second illegitimate child, shall be liable to imprisonment and hard
labour in Bridewell for six months.

Now as the number of females on this island much exceeds that of the
males; and as, from the mild temperature of the climate, circumstances
frequently arise among the lower ranks that render the intention of this
act of no effect; we could not help thinking this public exhibition of
the abstract as rather a rigorous exertion of Justice.

We found it was not very unusual here for the young men, from the
deficiency of numbers just spoken of, to pay their devoirs to more than
one young woman at a time; and as it is not possible for him legally to
unite himself to all of them, he generally bestows his hand on her who
had first presented him with a pledge of their love.

This, however, is seldom done till the approach of a second pledge from
the same person renders such an act of compassion needful, in order to
avoid the consequences of the tremendous anathema fixed on the church
door.
\end{quote}

You have been warned!


\backmatter


\end{document}
